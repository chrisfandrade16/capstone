\documentclass{article}

\usepackage{booktabs}
\usepackage{tabularx}

\title{Development Plan\\\progname}

\author{\authname}

\date{}

%% Comments

\usepackage{color}

\newif\ifcomments\commentstrue %displays comments
\newif\ifcomments\commentsfalse %so that comments do not display

\ifcomments
\newcommand{\authornote}[3]{\textcolor{#1}{[#3 ---#2]}}
\newcommand{\todo}[1]{\textcolor{red}{[TODO: #1]}}
\else
\newcommand{\authornote}[3]{}
\newcommand{\todo}[1]{}
\fi

\newcommand{\wss}[1]{\authornote{blue}{SS}{#1}} 
\newcommand{\plt}[1]{\authornote{magenta}{TPLT}{#1}} %For explanation of the template
\newcommand{\an}[1]{\authornote{cyan}{Author}{#1}}

%% Common Parts

\newcommand{\progname}{Sayyara} % PUT YOUR PROGRAM NAME HERE
\newcommand{\authname}{Team 31
\\ SFWRENG 4G06
\\ Christopher Andrade
\\ Alyssa Tunney
\\ Kai Zhu
\\ Ethan Vince-Budan
\\ Collin Kan
\\ Harsh Gupta} % AUTHOR NAMES                  

\usepackage{hyperref}
    \hypersetup{colorlinks=true, linkcolor=blue, citecolor=blue, filecolor=blue,
                urlcolor=blue, unicode=false}
    \urlstyle{same}

\begin{document}

\maketitle

\wss{Put your introductory blurb here.}

\newpage

\begin{table}[hp]
	\caption{Revision History} \label{TblRevisionHistory}
	\begin{tabularx}{\textwidth}{llX}
		\toprule
		\textbf{Date}      & \textbf{Developer(s)} & \textbf{Change}                                           \\
		\midrule
		September 24, 2022 & Alyssa Tunney         & Updated meeting plan, communication plan                  \\
		September 25, 2022 & Kai Zhu               & Added Workflow Plan                                       \\
		September 25, 2022 & Christopher Andrade   & Added Technology and Coding Standards joint section       \\
		September 25, 2022 & Collin Kan            & Added Proof of Concept Demonstration Plan                 \\
		September 26, 2022 & Harsh Gupta           & Added and edited risks, technology and project scheduling \\
		September 26, 2022 & Alyssa Tunney         & Modification of team member roles, misc. document cleanup \\
		September 26, 2022 & Collin Kan            & Added POC Demo Plan                                       \\
		September 26, 2022 & Kai Zhu               & Edit pass for consistency                                 \\
		\bottomrule
	\end{tabularx}
\end{table}

\newpage
\section{Team Meeting Plan}

The team will meet hold weekly in-person meetings every Friday at 2:30PM. The
agenda for each meeting will be determined  throughout the week based on
current and upcoming deliverable and development tasks. The meetings are
estimated to take between 30 minutes to 1 hour, with option to extend if
necessary. Where applicable, meeting minutes will be recorded by Alyssa Tunney.

\section{Team Communication Plan}

The primary method of internal team communication is through the Discord
application. Discord will also be used for asynchronous communication with the
team's supervisor, Nabeel Ibrahim. Online meetings with Nabeel will take place
on Google Meet. Development issues such as bugs and features, as well as
discussions regarding these topics will be tracked on the GitHub issues board
for traceability and ease of maintenance.

\section{Team Member Roles}

In addition to the roles below, each team member will be responsible for
contributing to the topics for meeting agendas during the team's weekly
meetings. Each team member is also expected to contribute to the documentation
and deliverable coursework.\\

\\\noindent Christopher Andrade
\begin{itemize}
	\item Team liaison, responsible for team communications such as emailing,
	      meeting scheduling, and meeting notes
	\item Testing lead
\end{itemize}
Alyssa Tunney
\begin{itemize}
	\item Responsible for recording meeting minutes
	\item Documentation lead
\end{itemize}
Kai Zhu
\begin{itemize}
	\item Back-end development lead
	\item Responsible for desktop application testing
\end{itemize}
Ethan Vince-Budan
\begin{itemize}
	\item Database development lead
	\item Responsible for mobile application testing
\end{itemize}
Collin Kan
\begin{itemize}
	\item System design lead
	\item Responsible for managing and organizing issues on GitHub
\end{itemize}
Harsh Gupta
\begin{itemize}
	\item Meeting chair
	\item Front-end development lead
\end{itemize}

\section{Workflow Plan}
The team will follow a feature branch workflow:

\begin{itemize}
	\item Issues are created for new features based on requirements using the
	      git issue board.
	\item Bug fixes, documentation-specific tasks such as proofreading,
	      sectional-editing are also categorized and tracked as issues.
	\item Issues are assigned to team members based on preference or expertise.
	\item For every new feature or bug fix, a new branch is created from a pull
	      of the latest main branch so that no change is pushed directly to the main
	      branch.
	\item The new feature or bug fix is implemented along with documentation
	      for each module and function.
	\item Commits should be small (such as after implementing a self-contained
	      function).
	\item Before a pull request, the branch should merge from the main branch
	      to resolve merge conflict.
	\item The developer(s) of the branch are responsible for performing unit
	      tests prior to merging.
	\item Pull requests are created upon the completion of the feature with
	      description of the changes, and linked to the corresponding issues for
	      clarity.
	\item Pull requests must pass automatic unit testing using GitHub actions
	      through CI/CD.
	\item Pull requests are reviewed and approved by another team member, and
	      the corresponding issues are marked as resolved.
\end{itemize}

\section{Proof of Concept Demonstration Plan}

\subsection{Main Risks}

\subsubsection{Implementation}
The primary difficulty of implementing this project is the scope of the
project. The MVP version of the web app consists of three main components, each
containing numerous different views and components. There is a concern that the
scope of the project potentially exceeds the amount of time available to
implement it. In addition, designing and implementing numerous architectural
components such as file and database structures, back-end API, and front-end
user interface add to the challenge of ensuring that all components work
together seamlessly.\\

\\Since the team's experience focus primarily in engineering rather than UX
design, creating an intuitive user interface may prove challenging, especially
when a clean, clutter-free, and navigable interface is one of the primary
requests from the client.

\subsubsection{Testing}
The application, as described by the client, contains multiple user types and
interactions. Consequently, it will likely contain a large number possible
states, such that creating and running unit test cases to cover every state of
the system can be challenging. In addition, as mentioned previously, important
but qualitative aspects of the system such as the intuitiveness of the user
interface are not measurable using automated tests. Manual testing of these
features may require outside users, which can be time consuming.\\

\\Furthermore, automating end-to-end tests over the numerous possible user
interactions will be a monumental undertaking. Additional care must also be
taken to managing and synchronizing the tests with feature delivery to
avoiding the risks of potentially breaking the CI/CD system.\\

\\The scalability of the system is a minor risk. During the development phase,
the client will only provide simulated data. Without a real active user base,
testing the system's performance and tolerance may be difficult.

\subsection{Risk Mitigation}

The main indicator of risk mitigation in the application will be the CI/CD
history and the number of consecutive successful deliveries to the main branch.
We expect to use Docker to further reduce the risk of dependency on specific
hardware and software environment for the development, testing, and deployment
stages.

\subsection{Demo Plan}

As a proof of concept, we will identify and implement at least one most
important feature for each business event in each of the three main components,
and ensure that the components interact and communicate with each other. The
primary focus will be on these core logic and communication, featuring a
minimal skeletal UI for demo purposes. This prototype will provide finer
insights into a realistic, achievable scope for the MVP. Upon successful
implementation of the POC, additional features can be added and the user
interface refined.


\section{Technology and Coding Standards}

The team will develope a full-stack application using a JavaScript front-end
and Python back-end. NextJS, using ReactJS as its base, will serve as the main
front-end framework to ensure web and mobile compatibility. The back-end will
use Supabase to provide a Postgres database, authentication, and instant APIs.
The StandardJS linter will be used to enforce front-end source code compliance
with the JavaScript Standard Style. For the Python back-end, PyLint will be
used to help follow the PEP8 coding standards. Unit testing will be conducted
using Jest and Pytest frameworks for Javascript and Python respectively. These
testing tools also have extensions to provide code coverage measurement with
LCOV reports. Using GitLab CI/CD, the project will also use Continuous
Integration with pipelines for testing and building before merge requests are
accepted to a master branch. Lastly, Docker may also be used to setup the
project for further building, testing, and deployment. Throughout the early
stages of development, some minor changes in the technologies may occur if
required.

\section{Project Scheduling}

\wss{How will the project be scheduled?}

The project schedule will be managed and tracked using GitHub issues, in
addition to online and in-person meetings. The project management feature in
GitHub will be used as the Kanban and brainstorming boards. Each requirement
will be decomposed into planing and implementation steps. The first major
milestone of the project is to create a functional web app setup such that
CI/CD can be deployed as a starting point for developers to deliver their
features. The other major milestones involve delivering specific features for
the given requirements, with the POC completed by November 15, 2022 and
revision 0 version delivered by February 6, 2023.

\end{document}